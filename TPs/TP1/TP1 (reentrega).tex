\documentclass[10pt,a4paper]{article}

\input{AEDmacros}
\usepackage{caratula} % Version modificada para usar las macros de algo1 de ~> https://github.com/bcardiff/dc-tex
\usepackage{mathtools}
\usepackage{amsfonts}
\usepackage{parskip}
\usepackage{indentfirst}
\usepackage{changepage}
\usepackage{amssymb}
\usepackage{xcolor}
\lstdefinestyle{base}{
  emptylines=1,
  breaklines=true,
  moredelim=**[is][\color{darkgray}]{@}{@},
  basicstyle=\ttfamily\footnotesize\bfseries,
  frame=tb,
}

\titulo{Trabajo Práctico 1}
\subtitulo{Especificación y WP}

\fecha{\today}

\materia{Algoritmos y Estructuras de Datos}
\grupo{Grupo ElDobleMenosUno}

\integrante{Deukmedjian, Iván}{521/24}{deukivan@gmail.com}
\integrante{Stescovich Curi, Agustín Ezequiel}{184/24}{agustinstescovichcuri@gmail.com}
\integrante{Feito, Agustín}{236/24}{agustinfeito@hotmail.com}
\integrante{Raffo, Pedro}{168/24}{pedroraffo25@gmail.com}
% Pongan cuantos integrantes quieran

% Declaramos donde van a estar las figuras
% No es obligatorio, pero suele ser comodo
\graphicspath{{../static/}}

\begin{document}

\maketitle

\section{Especificación}

\subsection{grandesCiudades}

% versión sin macros:

% \begin{tabbing}
%     tabula \=tabula \=3 \kill
%     \texttt{proc grandesCiudades} (in $ciudades : seq \langle \mathsf{Ciudad} \rangle) : seq \langle \mathsf{Ciudad} \rangle$ \\
%     \> \texttt{requiere} \{$\mathsf{noHayNombresRepetidos}(ciudades) \land \mathsf{todasPoblacionesPositivas}(ciudades)$\} \\
%     \> \texttt{asegura} \{$\mathsf{noHayNombresRepetidos}(res) \land ((c \in ciudades \land c.habitantes > 50000) \leftrightarrow c \in res)$\}
% \end{tabbing}

% \begin{tabbing}
%     tabula \=tabula \=3 \kill
%     \texttt{pred noHayNombresRepetidos} ($s : seq \langle \mathsf{Ciudad} \rangle$) \{ \\
%     \> $(\forall i : \mathbb{Z}) (0 \leq i < |s| \longrightarrow_L \lnot (\exists j : \mathbb{Z}) (0 \leq j < |s| \land_L i \neq j \land s[i].nombre = s[j].nombre))$ \\
%     \}  
% \end{tabbing}

% \begin{tabbing}
%     tabula \=tabula \=3 \kill
%     \texttt{pred todasPoblacionesPositivas} ($s : seq \langle \mathsf{Ciudad} \rangle$)
%     \{$(\forall i : \mathbb{Z}) (0 \leq i < |s| \longrightarrow_L s[i]_1 \geq 0)$\}
% \end{tabbing}

\begin{proc}{grandesCiudades}{\In ciudades : \TLista{\mathsf{Ciudad}}}{\TLista{\mathsf{Ciudad}}}
    \requiere{\mathsf{noHayNombresRepetidos}(ciudades) \land \mathsf{todasPoblacionesPositivas}(ciudades)}
    \asegura{\mathsf{noHayNombresRepetidos}(res) \land \paraTodo[unalinea]{c}{\mathsf{Ciudad}}{(c \in ciudades \land c.habitantes > 50000} \leftrightarrow c \in res)}
\end{proc}

\pred{noHayNombresRepetidos}{s : \TLista{\mathsf{Ciudad}}}
{\paraTodo[unalinea]{i}{\ent}{0 \leq i < |s| \longrightarrow_L \lnot \existe[unalinea]{j}{\ent}{0 \leq j < |s| \land_L i \neq j \land s[i].nombre = s[j].nombre}}}

\pred{todasPoblacionesPositivas}{s : \TLista{\mathsf{Ciudad}}}{\paraTodo[unalinea]{i}{\ent}{0 \leq i < |s| \longrightarrow_L s[i].habitantes \geq 0}}



\subsection{sumaDeHabitantes}

\textcolor{darkgray}{(\textsc{NOTA 1}: Para una mejor legibilidad de la especificación, se sustituye a $menoresDeCiudades$ por $men$ \\
y a $mayoresDeCiudades$ por $may$, es decir: $menoresDeCiudades = men$ y $mayoresDeCiudades = may$.)} \\
\textcolor{darkgray}{(\textsc{NOTA 2}: Reusamos aquí los predicados \textsf{noHayNombresRepetidos} y \textsf{todasPoblacionesPositivas}.)}

\begin{proc}{sumaDeHabitantes}{\In men, may : \TLista{\mathsf{Ciudad}}}{\TLista{\mathsf{Ciudad}}}
	\requiere{\mathsf{noHayNombresRepetidos}(men) \land \mathsf{noHayNombresRepetidos}(men) \land \mathsf{mismasCiudades}(may, \ men)}
	\requiere{\mathsf{todasPoblacionesPositivas}(may) \land \mathsf{todasPoblacionesPositivas}(may)}
	\asegura{\mathsf{noHayNombresRepetidos}(res) \land \mathsf{mismasCiudades}(res, men)}
	\begin{tabbing}
    tabula \=tabulaaa \=tabula \kill
    \> \texttt{asegura} \{$(\forall i, j : \ent) \ ((0 \leq i < |men| \land 0 \leq j < |may| \land_L men[i].nombre = may[j].nombre) \longrightarrow_L$ \\
    \> \> $(\exists k : \ent) \ (0 \leq k < |res| \land_L res[k].nombre = men[i].nombre$  \\
    \> \> $\land_L \ men[i].habitantes + may[j].habitantes = res[k].habitantes))$\}
    \end{tabbing}
    % sufrí demasiado haciendo que esto quede alineado
\end{proc}    

\pred{mismasCiudades}{c1, c2 : \TLista{\mathsf{Ciudad}}}
{|c1| = |c2| \land_L \paraTodo[unalinea]{i}{\ent}{0 \leq i < |c1| \longrightarrow_L \existe[unalinea]{j}{\ent}{0 \leq j < |c2| \land_L c1[i].nombre = c2[j].nombre}}}
% si ambas tienen la misma longitud y cada elemento es distinto, pedir esto es que ambas listas estén emparejadas uno-a-uno



\subsection{hayCamino}

% versión sin macros:

% \begin{tabbing}
%     tabula \=tabula \=tabula \kill
%     \texttt{proc hayCamino} (in $distancias: seq \langle seq \langle \mathbb{Z} \rangle \rangle,$ in $desde, hasta : \mathbb{Z}) : \mathsf{Bool}$ \\
%     \> \texttt{requiere} \{$\mathsf{esMatrizCuadrada}(distancias) \land \mathsf{esMatrizSimetrica}(distancias)$\} \\
%     \> \texttt{requiere} \{$\mathsf{ningunElementoNegativo}(distancias) \land 0 \leq desde, hasta < |distancias|$\} \\
%     \> \texttt{asegura} \{$res = \mathtt{true} \leftrightarrow (\exists i : seq \langle \mathbb{Z} \rangle) (\mathsf{esCaminoValido}(i, desde, hasta, distancias)$\}
% \end{tabbing}
    
% \begin{tabbing}
%     tabula \=tabula \=tabula \kill
%     \texttt{pred esMatrizCuadrada} $(m : seq \langle seq \langle \mathbb{Z} \rangle \rangle)$ \{ \\
%     \> $(\forall i : \mathbb{Z}) (0 \leq i < |m| \longrightarrow_L |m| = |m[i]|)$\}
% \end{tabbing}
    
% \begin{tabbing}
%     tabula \=tabula \=tabula \kill 
%     \texttt{pred esMatrizSimetrica} $(m : seq \langle seq \langle \mathbb{Z} \rangle \rangle)$ \{ \\
%     \> $(\forall i, j : \mathbb{Z}) (0 \leq i, j < |m| \longrightarrow_L m[i][j] = m[j][i])$\}
% \end{tabbing}
    
% \begin{tabbing}
%     tabula \=tabula \=tabula \kill
%     \texttt{pred ningunElementoNegativo} $(s : seq \langle seq \langle \mathbb{Z} \rangle \rangle)$ \{ \\
%     \> $(\forall i, j : \mathbb{Z}) (0 \leq i, j < |m| \longrightarrow_L m[i][j] \geq 0)$\}
% \end{tabbing}

% \begin{tabbing}
%     tabula \=tabula \=tabula \kill 
%     \texttt{pred esCaminoValido} $(camino : seq \langle \mathbb{Z} \rangle, origen : \mathbb{Z}, destino : \mathbb{Z}, m : seq \langle \langle \mathbb{Z} \rangle \rangle )$ \{ \\
%     \> $|camino| \geq 2 \land_L camino[0] = origen \land camino[|camino| - 1] = destino$ \\
%     \> $\land (\forall i : \mathbb{Z}) (0 \leq i < |camino| - 1 \longrightarrow_L (camino[i] \neq camino[i + 1] \rightarrow m[i][i + 1] > 0))$\}
% \end{tabbing}

\begin{proc}{hayCamino}{\In distancias : \TLista{\TLista{\ent}}, \In desde, hasta : \ent}{\textsf{Bool}}
    \requiere{\mathsf{esMatrizCuadrada}(distancias) \land \mathsf{esMatrizSimetrica}(distancias)}
    \requiere{\mathsf{ningunElementoNegativo}(distancias) \land 0 \leq desde, hasta < |distancias|}
    \asegura{res = \True \leftrightarrow \existe[unalinea]{i}{\TLista{\ent}}{\mathsf{esCaminoValido}(i, desde, hasta, distancias)}}
\end{proc}

\texttt{pred esMatrizCuadrada} $(m : seq \langle seq \langle \mathbb{Z} \rangle \rangle)$ \{$(\forall i : \mathbb{Z})$ $(0 \leq i < |m| \longrightarrow_L |m| = |m[i]|)$\}

\texttt{pred esMatrizSimetrica} $(m : seq \langle seq \langle \mathbb{Z} \rangle \rangle)$ \{$(\forall i, j : \mathbb{Z})$ $(0 \leq i, j < |m| \longrightarrow_L m[i][j] = m[j][i])$\}
    
\texttt{pred ningunElementoNegativo} $(s : seq \langle seq \langle \mathbb{Z} \rangle \rangle)$ \{$(\forall i, j : \mathbb{Z})$ $(0 \leq i, j < |m| \longrightarrow_L m[i][j] \geq 0)$\}
% estos los paso sin macros para que queden en una sola línea - si no ocupan mucho espacio

% \pred{esMatrizCuadrada}{m : \TLista{\TLista{\ent}}}{\paraTodo[unalinea]{i}{\ent}{0 \leq i < |m| \longrightarrow_L |m| = |m[i]|}}
% \pred{esMatrizSimetrica}{m : \TLista{\TLista{\ent}}}{\paraTodo[unalinea]{i, j}{\ent}{0 \leq i, j < |m| \longrightarrow_L m[i][j] = m[j][i]}}
% \pred{ningunElementoNegativo}{m : \TLista{\TLista{\ent}}}{\paraTodo[unalinea]{i, j}{\ent}{0 \leq i, j < |m| \longrightarrow_L m[i][j] \geq 0}}

\pred{esCaminoValido}{camino : \TLista{\ent}, origen, destino : \ent, m : \TLista{\TLista{\ent}}}
{|camino| \geq 2 \land_L (camino[0] = origen \land camino[|camino| - 1] = destino) \land_L \\
\paraTodo[unalinea]{i}{\ent}{0 \leq i < |camino| - 1 \longrightarrow_L m[camino[i]][camino[i + 1]] > 0}}



\subsection{cantidadCaminosNSaltos}

\textcolor{darkgray}{(\textsc{NOTA}: Reusamos aquí los predicados \textsf{esMatrizCuadrada} y \textsf{esMatrizSimetrica}.)}

\begin{proc}{cantidadCaminosNSaltos}{\Inout conexion : \TLista{\TLista{\ent}}, \In n : \ent}{}
	\requiere{conexion = C_0 \land \mathsf{esMatrizCuadrada}(conexion) \land \mathsf{esMatrizSimetrica}(conexion) \land n \geq 1}
    \requiere{\paraTodo[unalinea]{i,j}{\ent}{0 \leq i, j < |conexion| \longrightarrow_L 0 \leq conexion[i][j] \leq 1}}
	\asegura{\mathsf{esMatrizCuadrada}(conexion) \land \mathsf{esMatrizSimetrica}(conexion) \land \mathsf{esNEsimaPotencia}(conexion, \ C_0, \ n)}
\end{proc}

\pred{esNEsimaPotencia}{matriz, base : \TLista{\TLista{\ent}}, exp : \ent}
{\existe[unalinea]{s}{\TLista{\TLista{\TLista{\ent}}}}{|s| = exp \land_L s[0] = base \land s[exp - 1] = matriz \land_L \\
\paraTodo[unalinea]{i}{\ent}{0 \leq i < exp \longrightarrow_L |s[i]| = |s[0]| \land \mathsf{esMatrizCuadrada}(s[i]) \land_L \\
(i \neq 0 \longrightarrow_L \mathsf{esProducto}(s[i], \ s[i-1], \ s[0]))}}}


\pred{esProducto}{matriz, factor1, factor2 : \TLista{\TLista{\ent}}}
{\paraTodo[unalinea]{i, j}{\ent}{0 \leq i, j < |matriz| \longrightarrow_L 
matriz[i][j]=\sum_{k = 0}^{|matriz| - 1} factor1[i][k] * factor2[k][j]}}


\subsection{caminoMinimo}

% versión sin macros:

% \begin{tabbing}
%     tabula \=tabula \=tabula \kill 
%     \texttt{proc caminoMinimo} $(origen : \mathbb{Z}, destino : \mathbb{Z}, distancias : seq \langle \langle \mathbb{Z} \rangle \rangle ) : seq \langle \mathbb{Z} \rangle $ \{ \\
%     \> \texttt{requiere} \{$0 \leq origen, destino < |distancias| \land \mathsf{esMatrizCuadrada}(distancias) \land \mathsf{esMatrizSimetrica}(distancias)$\} \\
%     \> \texttt{asegura} \{$res = \leftrightarrow \lnot (\exists i : seq \langle \mathbb{Z} \rangle) (\mathsf{esCaminoValido}(i, origen, destino, distancias)$\} \\
%     \> \texttt{asegura} \{$res \neq \langle \rangle \leftrightarrow \mathsf{esCaminoMinimo}(res, origen, destino, distancias)$\}
%     \end{tabbing}

% \begin{tabbing}
%     tabula \=tabula \=tabula \kill 
%     \texttt{aux distanciaCamino} $(camino : seq \langle \mathbb{Z} \rangle, m : seq \langle \langle \mathbb{Z} \rangle \rangle) = \sum_{i = 0}^{|camino| - 1} m[i][i + 1]$
% \end{tabbing}

% \begin{tabbing}
%     tabula \=tabula \=tabula \kill 
%     \texttt{pred esCaminoMinimo} $(camino : seq \langle \mathbb{Z} \rangle, origen : \mathbb{Z}, destino : \mathbb{Z}, m : seq \langle \langle \mathbb{Z} \rangle \rangle )$ \{ \\
%     \> $\mathsf{esCaminoValido}(res, origen, destino, m) \land_L$ \\
%     \> $(\forall i : seq \langle \mathbb{Z} \rangle) (\mathsf{esCaminoValido}(i, origen, destino, m) \longrightarrow_L \mathsf{distanciaCamino}(i, m) \geq \mathsf{distanciaCamino}(res, m))$\}
% \end{tabbing}

\textcolor{darkgray}{(\textsc{NOTA}: Reusamos aquí los predicados \textsf{esMatrizCuadrada}, \textsf{esMatrizSimetrica} y \textsf{ningunElementoNegativo}.)}

\begin{proc}{caminoMinimo}{origen, destino : \ent, distancias : \TLista{\TLista{\ent}}}{\TLista{\ent}}
    \requiere{\mathsf{esMatrizCuadrada}(distancias) \land \mathsf{esMatrizSimetrica}(distancias)}
    \requiere{\mathsf{ningunElementoNegativo}(distancias) \land 0 \leq origen, \ destino < |distancias|}
    \asegura{res = \langle \rangle \leftrightarrow \lnot \existe[unalinea]{i}{\TLista{\ent}}{\mathsf{esCaminoValido}(i, \ origen, \ destino, \ distancias)}}
    \asegura{res \neq \langle \rangle \leftrightarrow \mathsf{esCaminoMinimo}(res, \ origen, \ destino, \ distancias)}
\end{proc}

\aux{distanciaCamino}{camino : \TLista{\ent}, m : \TLista{\TLista{\ent}}}{\ent}{\sum\limits_{i = 0}^{|camino| - 1} m[camino[i]][camino[i + 1]]}

\pred{esCaminoMinimo}{camino : \TLista{\ent}, origen, destino : \ent, m : \TLista{\TLista{\ent}}}
{\mathsf{esCaminoValido}(res, \ origen, \ destino, \ m) \land_L \paraTodo[unalinea]{i}{\TLista{\ent}}{ \\
\mathsf{esCaminoValido}(i, \ origen, \ destino, \ m) \longrightarrow_L \mathsf{distanciaCamino}(i, \ m) \geq \mathsf{distanciaCamino}(res, \ m)}}



\newpage



\section{Correctitud}
\subsection{Demostración de correctitud de la implementación de \textsf{poblaciónTotal}}

Esta es la tripla de Hoare a probar:
\begin{equation*}
    \begin{split}
        &\textbf{P} \equiv \{(\exists i:\mathbb{Z})(0 \leq i < |ciudades| \land_L ciudades[i].habitantes > 50,000) \land \\
        &(\forall i : \mathbb{Z})(0 \leq i < |ciudades| \longrightarrow_L ciudades[i].habitantes \geq 0) \land\\
        &(\forall i :\mathbb{Z})(\forall j \mathbb{Z})(0 \leq i < j < |ciudades| \longrightarrow_L ciudades[i].nombre \ne ciudades[j].nombre)\}
    \end{split}
\end{equation*}
\begin{lstlisting}[style=base]
    res := 0;
    i := 0;
    while (i < ciudades.length) do 
        res = res + ciudades[i].habitantes; @Nuestro S1 en el ciclo@
        i := i + 1; @Nuestro S2 en el ciclo@
    endwhile
\end{lstlisting}
\begin{equation*}
    \begin{split}
        &\textbf{Q}\equiv \{res = \sum_{j = 0}^{|ciudades| - 1} ciudades[j].habitantes\}
    \end{split}
\end{equation*}


\subsubsection{Teorema del invariante}
Proponemos el siguiente invariante y demostramos la correctitud parcial del ciclo: 
\begin{equation*}
    I \equiv \{0 \leq i \leq |ciudades| \land_L res = \sum\limits_{j = 0}^{i - 1} ciudades[j].habitantes\}
\end{equation*}

\begin{itemize}
\item{\textbf{Condición 1:} $\mathbf{P_c \Rightarrow I}$} \\
\end{itemize}
\begin{equation*}
    \begin{split}
        P_c \equiv \ 
        &(\exists i:\mathbb{Z})(0 \leq i < |ciudades| \land_L ciudades[i].habitantes > 50,000) \ \land \\
        & (\forall i : \mathbb{Z})(0 \leq i < |ciudades| \longrightarrow_L ciudades[i].habitantes \geq 0) \ \land \\
        &(\forall i :\mathbb{Z})(\forall j : \mathbb{Z})(0 \leq i < j < |ciudades| \longrightarrow_L ciudades[i].nombre \ne ciudades[j].nombre) \ \land \\
        & res = 0 \land i = 0 \Longrightarrow (0 \leq i \leq |ciudades| \land res = \sum\limits_{j = 0} ^ {i - 1} ciudades[j].habitantes)
    \end{split}
\end{equation*}

Utilizamos que $P_c \Rightarrow res = 0 \land i = 0$ y reemplazamos en el invariante:
        \begin{equation*}
            \begin{split}
                & \ 0 \leq 0 \leq |ciudades| \land res = \sum\limits_{j = 0} ^ {-1}Ciudades[j].habitantes \\
                \equiv & \ True \land res = 0 \\
                \equiv & \ res = 0
            \end{split}
        \end{equation*}
Se evidencia que $P_c \Rightarrow res = 0 \land i = 0 \Rightarrow res = 0$, y $\therefore$ se cumple que el antecedente implica al consecuente. \\


\begin{itemize}
    \item{\textbf{Condición 2: validez de la tripla} $\mathbf{\{I \land B\} \ \texttt{S} \ \{I\}}$}
\end{itemize}

Comenzamos resolviendo la WP:
    \setcounter{equation}{0}
        \begin{align}
                \textbf{wp(\texttt{S}, I)} &\equiv wp(\texttt{S1; S2}, \ I) \\
                &\equiv wp(\texttt{S1}, \ wp(\texttt{S2}, \ I)) \\
                &\equiv wp(\texttt{S1}, \ wp(\texttt{i := i + 1}, \ (0 \leq i \leq |ciudades| \land res = \sum\limits_{j = 0}^{i - 1} ciudades[j].habitantes)) \\
                &\equiv wp(\texttt{S1}, \ 0 \leq i + 1 \leq |ciudades| \land res = \sum\limits_{j = 0}^{i} ciudades[j].habitantes) \\
                &\equiv (0 \leq i + 1 \leq |ciudades| \land 0\leq i < |ciudades| \ \land \\
                & \ \ \ \ res + ciudades[i].habitantes = (\sum\limits_{j = 0}^{i - 1} ciudades[j].habitantes) + ciudades[i].habitantes \\
                &\equiv 0 \leq i < |ciudades| \land res = \sum\limits_{j = 0}^{i - 1} ciudades[j].habitantes
        \end{align}

Ahora vemos la implicación $\mathbf{(I\land B) \Rightarrow wp(\textbf{S},I)}$:
    \setcounter{equation}{0}
    \begin{align}
               & \equiv 0 \leq i \leq |ciudades| \land res = \sum\limits_{j = 0} ^ {i - 1} ciudades[j].habitantes \land i < |ciudades| \\
               & \equiv 0 \leq i < |ciudades| \land res = \sum\limits_{j = 0} ^ {i - 1} ciudades[j].habitantes \\
               \nonumber & \Rightarrow 0 \leq i < |ciudades| \land res =\sum\limits_{j=0} ^ {i-1} ciudades[j].habitantes 
    \end{align}

Esto se cumple dado que es una tautología del tipo $p \Rightarrow p$. \\


\begin{itemize}
    \item{\textbf{Condición 3:} $\mathbf{I \land \lnot B \Rightarrow Q_c}$}
\end{itemize}

\setcounter{equation}{0}
\begin{align}
        \nonumber & \ \ \ \ I \land \lnot B \\ 
        & \equiv 0 \leq i \leq |ciudades| \land_L res = \sum_{j = 0}^{i - 1} ciudades[j].habitantes \land \lnot (i < |ciudades|) \\
        & \equiv i = |ciudades| \land_L res = \sum_{j = 0}^{i - 1} ciudades[j].habitantes \\
        & \equiv i = |ciudades| \land_L res = \sum_{j = 0}^{|ciudades| - 1} ciudades[j].habitantes \\
        \nonumber & \Rightarrow res = \sum_{j = 0}^{|ciudades| - 1} ciudades[j].habitantes \\
        \nonumber & \equiv True
\end{align}

\ldots pues $p \land q \Rightarrow q$ es tautología.



\subsubsection{Teorema de terminación}
Proponemos la función variante $fv = |ciudades| - i$ y buscamos probar que el ciclo satisface el teorema de terminación. \\

\begin{itemize}
    \item{\textbf{Condición 4: validez de la tripla} $\mathbf{\{I \land B \land fv = v_0\} \ \texttt{s} \ \{fv < v_0\}}$}
\end{itemize}
Primero, debemos calcular la precondición más débil del ciclo respecto de $fv < v_0$:
\begin{equation*} 
    \begin{split}
    & wp(\mathtt{res := res + ciudades[i].habitantes; \ i := i + 1}, \ fv < v_0) \\
    \equiv \ & wp(\mathtt{res := res + ciudades[i].habitantes}, \ (wp(\mathtt{i := i + 1}, \ fv < v_0))
    \end{split}
\end{equation*}

Calculemos primero la WP anidada:
\begin{equation*}
    \begin{split}
    & wp(\mathtt{i := i + 1}, \ fv < v_0) \\
    \equiv \ & wp(\mathtt{i := i + 1}, \ |ciudades| - i < v_0) \\
    \equiv \ & def(i + 1) \land_L Q_{i + 1}^{i} \\
    \equiv \ & True \land_L |ciudades| - i - 1 < v_0 \\
    \equiv \ & |ciudades| - i - 1 < v_0
    \end{split}
\end{equation*}

Usando esto,
\begin{equation*}
    \begin{split}
    & wp(\mathtt{res := res + ciudades[i].habitantes}, \ (wp(\mathtt{i := i + 1}, \ fv < v_0)) \\
    \equiv \ & wp(\mathtt{res := res + ciudades[i].habitantes}, \ |ciudades| - i - 1 < v_0) \\
    \equiv \ & def(res + ciudades[i].habitantes) \land_L Q_{res \ + \ ciudades[i].habitantes}^{res} \\
    \equiv \ & 0 \leq i < |ciudades| \land_L |ciudades| - i - 1 < v_0 \\
    \end{split}
\end{equation*}

Dada la precondición más débil, comprobamos ahora que $I \land B \land fv = v_0$ la fuerza:
\begin{equation*}
    \begin{split}
    & I \land B \land fv = v_0 \\
    \equiv \ & 0 \leq i \leq |ciudades| \land_L res = \sum_{j = 0}^{i - 1} ciudades[j].habitantes \land i < |ciudades| \land |ciudades| - i = v_0
    \end{split}
\end{equation*}

Basta ver que $|ciudades| - i = v_0 \Rightarrow |ciudades| - i - 1 < v_0$, pues 
\begin{equation*}
    \begin{split}
    & |ciudades| - i - 1 < |ciudades| - i \\
    \equiv \ & -1 < 0 \\
    \equiv \ & True
    \end{split}
\end{equation*}
Por lo tanto, la tripla de Hoare $\{I \land B \land fv = v_0\} \ \mathtt{s} \ \{fv < v_0\}$ es válida. \\


\begin{itemize}
    \item{\textbf{Condición 5:} $\mathbf{(I \land fv \leq 0) \Rightarrow \lnot B}$}
\end{itemize}

Ahora, debemos mostrar que la guarda del ciclo deja de valer cuando la función variante es menor o igual a 0. \\
\begin{equation*}
    \begin{split}
    & I \land fv \leq 0 \\
    \equiv \ & 0 \leq i \leq |ciudades| \land_L res = \sum_{j = 0}^{i - 1} ciudades[j].habitantes \land |ciudades| - i < 0
    \end{split}
\end{equation*}

Tenemos que\ldots \\
\begin{equation*}
    \begin{split}
        & |ciudades| - i < 0 \land 0 \leq i \leq |ciudades| \\
        \Rightarrow \ & i = |ciudades| \\
        \Rightarrow \ & \lnot(i < |ciudades|)
    \end{split}
\end{equation*}
\ldots que es la implicación que buscábamos probar.



\subsubsection{Finalización de la demostración}
Sabemos que el ciclo del programa es correcto respecto de $(P_c, \ Q_c)$, pero falta demostrar la validez de \\
$\{P\} \ \mathtt{res := 0; i := 0} \ \{P_c\}$, es decir, que se llegue siempre a la precondición del ciclo partiendo de la especificación. \\

Comenzamos calculando la precondición más debil de las instrucciones anteriores al ciclo respecto de $P_c$:
\begin{equation*}
    \begin{split}
    & wp(\mathtt{res := 0; \ i := 0}, \ P_c) \\
    \equiv \ & wp(\mathtt{res := 0}, \ wp(\mathtt{i := 0}, \ P_c))
    \end{split}
\end{equation*}

\textcolor{darkgray}{(\textsc{NOTA}: para ahorrar espacio, recordamos que $P_c \equiv P \land res = 0 \land i = 0$.)}\\
\begin{equation*}
    \begin{split}
    & wp(\mathtt{i := 0}, \ P_c)) \\
    \equiv \ & def(0) \land_L Q_{0}^{i} \\
    \equiv \ & True \land_L (P \land res = 0 \land 0 = 0) \\
    \equiv \ & P \land res = 0
    \end{split}
\end{equation*}

Entonces, \\
\begin{equation*}
    \begin{split}
    & wp(\mathtt{res := 0}, \ wp(\mathtt{i := 0}, \ P_c)) \\
    \equiv \ & wp(\mathtt{res := 0}, \ P \land res = 0) \\
    \equiv \ & def(0) \land_L Q_{0}^{res} \\
    \equiv \ & P \land 0 = 0 \\
    \equiv \ & P \\
    \end{split}
\end{equation*}

Se sigue trivialmente que $P \Rightarrow P$, y por lo tanto la tripla $\{P\} \ \mathtt{res := 0; i := 0} \ \{P_c\}$ es válida. \\
Luego, por monotonía, como $\{P\} \ \mathtt{res := 0; i := 0} \ \{P_c\}$ y $\{P_c\} \ \mathtt{S} \ \{Q\}$ son válidas, también lo es $\{P\} \ \mathtt{res := 0; i := 0; S} \ \{Q\}$. \\
Así, hemos demostrado que la implementación de \textsf{poblaciónTotal} es correcta respecto de su especificación.

\newpage

\subsection{¿El valor devuelto por el programa es mayor a 50.000?}

Demostramos anteriormente que la implementación \texttt{S} es correcta respecto de la especificación, y por tanto basta ver que la especificación cumple lo pedido.

Sabemos  que \textit{ciudades} es un parámetro de tipo \textsf{in}, por lo que no se modifica en ningún punto del programa. \\
Además, obtenemos de la precondición del programa que\ldots

\begin{itemize}
    \item{\existe[unalinea]{i}{\ent}{0 \leq i < |ciudades| \land_L ciudades[i].habitantes > 50,000} \\
          (existe al menos un elemento de \textit{ciudades} cuya segunda componente es estrictamente mayor a $50,000$),}
    \item{\paraTodo[unalinea]{i}{\ent}{0 \leq i < |ciudades \longrightarrow_L ciudades[i].habitantes \geq 0} \\
          (ninguna segunda componente de un elemento en \textit{ciudades} es negativa),}

\ldots y de la postcondición que\ldots

    \item{$res = \sum\limits_{i = 0}^{|ciudades| - 1} ciudades[i].habitantes$ \\
          (el resultado obtenido es la sumatoria de la segunda componente de todo elemento en \textit{ciudades}).}
\end{itemize}

Por tanto, dadas estas condiciones, se garantiza que $\sum\limits_{j = 0}^{|ciudades| - 1} ciudades[j].habitantes$ sea mayor a $50.000$.

\end{document}
